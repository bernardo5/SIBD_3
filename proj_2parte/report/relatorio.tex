\documentclass[a4paper]{article}

\usepackage[portuguese]{babel}
\usepackage{comment}
\usepackage[T1]{fontenc}
\usepackage[utf8]{inputenc}
\usepackage{hyperref}
\usepackage{graphicx}
\usepackage{float}
\usepackage{multirow}
\usepackage{indentfirst}
\usepackage{listings}
\usepackage[hypcap]{caption} % makes \ref point to top of figures and tables
\newcommand{\tab}[1]{\hspace{.2\textwidth}\rlap{#1}}
\usepackage[]{algorithm2e}
\usepackage{tabularx}
\usepackage{lscape}

\begin{document}
\lstset{language=Pascal}
\begin{titlepage}

	\begin{center}

		\includegraphics[width=6cm]{./title}\\[3cm]

		\textsc{\LARGE Sistemas de Informação e Bases de Dados}\\[1.5cm]

		\textsc{\Large 2ª Parte do Projeto}\\[1.5cm]
		
		\textsc{\Large Grupo 13}\\[1.5cm]


		


		\noindent
		\begin{minipage}{0.4\textwidth}
			\begin{flushleft} \large
				Diogo Proença, 75313
			\end{flushleft}
		\end{minipage}
		\begin{minipage}{0.4\textwidth}
			\begin{flushright} \large
				Diogo Martins, 75462
			\end{flushright}
		\end{minipage}
		
		\begin{minipage}{0.4\textwidth}
			\begin{flushright} \large
				Bernardo Gomes, 75573	
			\end{flushright}
		\end{minipage}

		\vfill

		{\large \today}


	\end{center}

\end{titlepage}
\hypersetup{%
    pdfborder = {0 0 0}
}

\tableofcontents
\pagenumbering{gobble}
\pagebreak

\pagenumbering{arabic}
\section{Criação das tabelas da base de dados}
Para a criação das tabelas na base de dados, as intruções de SQL utilizadas foram as seguintes:

\lstinputlisting{database.sql}

Relativamente às escolhas das variáveis o seguinte conjunto merece especial destaque:
\begin{itemize}

  \item \textbf{variável \textit{number} da tabela \textit{Patient}} $\rightarrow$ ao termos levado a cabo pesquisa relativa a \textit{Social Security Numbers} (SSN's) válidos, verificámos que este conjunto de números pode conter caracteres não numéricos ("-"). Além deste facto, pode também começar pelo número "0", o que invalida o uso de variáveis \textit{integer}, \textit{numeric} e \textit{decimal} caso contrário o número armazenado iria perder dígitos. Utiliza-se assim, a variável \textit{varchar};
  
  \item \textbf{variável \textit{phone} da tabela \textit{PAN}} $\rightarrow$ tendo em conta a existência de números de telefone com prefixo diferente em cada país (\textit{e.g.} Portugal - +351), e não sendo necessário que o paciente tenha um número do país onde o \textit{Medical Center} está localizado, utiliza-se a variável \textit{varchar};
  
  \item \textbf{variável \textit{serialnum} da tabela \textit{Device}} $\rightarrow$ utiliza-se a variável \textit{integer}. No entanto, poder-se-ia considerar também do tipo \textit{varchar} pelo facto de em alguns casos os números de série conterem caracteres. No entanto, não tendo sendo especificado nada no enunciado, e não tendo obtido nenhuma informação sobre os números de série de aparelhos médicos, considera-se o caso em que estes podem ser representados por um número;
  
  \item Para as tabelas \textit{Sensor} e \textit{Actuator} é utilizado o mesmo critério que a tabela anterior;
  
  \item \textbf{variável \textit{nut4code} da tabela \textit{Municipality}} $\rightarrow$ para esta tabela, considera-se apenas os códigos postais semelhantes a Portugal, identificados por quatro números, tal como o nome da variável indica;
  
  \item \textbf{variáveis \textit{start} e \textit{end} da tabela \textit{Period}} $\rightarrow$ escolhe-se o tipo \textit{date} pelo facto de as relações com esta entidade não necessitar de precisões ao nível HH:mm:ss;
  
  \item para as tabelas \textit{Period}, \textit{Wears}, \textit{Lives} e \textit{Connects} foi colocada uma verificação das datas \textit{check(start<=end)} de forma a garantir que o período está consistente;

  \item \textbf{variável \textit{datetime} das tabelas \textit{Reading} e \textit{Setting}} $\rightarrow$ escolhe-se o tipo \textit{timestamp} pelo facto de as medições médicas necessitarem de precisões ao nível HH:mm:ss;
  
  \item \textbf{variável \textit{value} das tabelas \textit{Reading} e \textit{Setting}} $\rightarrow$ escolhe-se o tipo \textit{numeric} pelo facto de permitir precisão ao nível de casas decimais.  
\end{itemize}
\pagebreak
Inicialmente, acrescentam-se as seguintes instruções de forma a apagar eventuais tabelas com o mesmo nome antes da criação das novas:

\lstinputlisting{drop.sql}

\section{\textit{Triggers} para prevenção de \textit{overlapping periods}}
A função dos \textit{triggers} é de prevenir a inserção/actualização de associações a PANs por parte de pacientes ou de aparelhos médicos em periodos de tempo sobrepostos. 

A \textit{error message} deve ser apresentada de forma a evitar que ocorram casos,  para a tabela \textit{Wears}, como os seguintes:
\vskip 5mm
\begin{itemize}
\item Paciente 1 está ligado a PAN1 e PAN2 ao mesmo tempo;

\item  Paciente 1 e Paciente 2 estão ligados à PAN1 ao mesmo tempo.
\end{itemize}

A \textit{error message} deve ser apresentada de forma a evitar que ocorram casos,  para a tabela \textit{Connects}, como o seguinte: 

\begin{itemize}

\item Device 1 está ligado a PAN1 e PAN2 ao mesmo tempo.

\end{itemize}

A condição de sobreposição de períodos, independentemente da tabela em questão obtém-se através da análise do problema com a aplicação das leis de De Morgan. Ora, não existe sobreposição de períodos em dois casos:
\begin{enumerate}
\item o intervalo inserido/atualizado é completamente anterior aos intervalos já inseridos na tabela;
\item o intervalo inserido/atualizado é completamente posterior aos intervalos já inseridos na tabela.
\end{enumerate}

Seja A um intervalo definido por startA e endA e o intervalo B definido por startB e endB. Os intervalos A e B não se encontram sobrepostos sempre que: (startA$\geq$endB) $\cup$ (endA$\leq$startB).

Negando a condição obtém-se todos os casos para os quais os intervalos A e B se encontram sobrepostos. Assim, através das leis de De Morgan obtem-se a seguinte condição para dois intervalos sobrepostos: (startA$\leq$endB)$\cap$(endA$\geq$startB). Esta condição é aplicada em todos os \textit{triggers} que se apresentam de seguida.

\subsection{\textit{Triggers} de \textit{insert}}
Para a tabela \textit{Connects}, não se pode inserir o mesmo \textit{device} em duas PANs distintas no mesmo período de tempo. Desta forma, para cada linha dentro da tabela onde se pretende inserir o novo aparelho, será avaliado \textit{a priori} se este já se encontra contido nos registos da tabela. Posteriormente, avalia-se se este \textit{device} possui um PAN diferente, associado em períodos de tempo sobrepostos aos que se encontram nos registos da tabela.

O \textit{trigger} para a tabela \textit{Connects} tem o seguinte código:

\lstinputlisting{trigger_pan_device_insert.sql}

Relativamente à tabela \textit{Wears}, o raciocínio aplicado é semelhante. Um paciente não pode ter duas PANs ao mesmo tempo e dois pacientes não podem partilhar a mesma PAN. Este caso, gera uma condição adicional, que não surgia no caso anterior. As condições são assim semelhantes às apresentadas anteriormente, sendo ainda verificado se o paciente se encontra na tabela e se a PAN que se pretende que o mesmo utilize está actualmente em utilização por um paciente distinto. No caso de o paciente se encontrar nos registos, verifica-se se a PAN que está a utilizar é a mesma e única no período de tempo.
\pagebreak

O \textit{trigger} para a tabela \textit{Wears} tem o seguinte código:
\vskip 5mm
\lstinputlisting{trigger_pan_patient_insert.sql}
%----------------------------------------------------------------------------------------------
\vskip 5mm
Por forma a testar o funcionamento dos \textit{triggers} atrás apresentados, utilizámos a base de dados fornecida no ficheiro \textit{database\_triggers\_insert.sql}. Após correr as instruções, obtêm-se as seguintes tabelas:

\begin{figure}[ht!]
\centering
\includegraphics[scale=0.7]{tabelas_insert_triggers_Connects.png}
\caption{Tabela \textit{Connects}}
\end{figure}

\begin{figure}[ht!]
\centering
\includegraphics[scale=0.7]{tabelas_insert_triggers_Wears.png}
\caption{Tabela \textit{Wears}}
\end{figure}

\vskip 7mm
Os testes realizados foram sucessivas inserções de um dispositivo em PANs diferentes sobreposto a pelo menos uma entrada da tabela.

Os testes realizados estão contidos nos ficheiros \textit{teste\_insert\_connects.sql} e \textit{teste\_insert\_connects.sql}, apresentados em anexo e que contem uma série de instruções e respectivos comentários.

Correndo os ficheiros de teste verifica-se um de quatro dos seguintes resultados, consoante as tabelas a alterar e cosoante seja uma um \textit{update}/inserção válido ou inválido:

\begin{figure}[ht!]
\centering
\includegraphics[scale=0.7]{erro_connect.png}
\caption{Mensagem de erro de inserção para a tabela \textit{Connects}}
\end{figure}

\begin{figure}[ht!]
\centering
\includegraphics[scale=0.7]{good_connect.png}
\caption{Mensagem sucesso de inserção para a tabela \textit{Connects}}
\end{figure}

\begin{figure}[ht!]
\centering
\includegraphics[scale=0.53]{error_wears.jpg}
\caption{Resultados dos testes do \textit{trigger} de inserção para a tabela \textit{Connects}}
\end{figure}

\begin{figure}[ht!]
\centering
\includegraphics[scale=0.53]{good_wears.jpg}
\caption{Resultados dos testes do \textit{trigger} de inserção para a tabela \textit{Connects}}
\end{figure}

Em caso de sucesso, ou seja, sem sobreposição, surge uma mensagem de erro devido ao facto de o período inserido não existir na tabela \textit{Period}. Caso estivesse, não ocorreria o erro e a tabela seria atualizada.
\pagebreak
\subsection{\textit{Triggers} de \textit{update}}
À semelhança dos testes que realizámos para os \textit{triggers} de \textit{insert} para as duas tabelas, os ficheiros de teste para os \textit{triggers} de \textit{update} encontram-se no ficheiro .zip, com os nomes \textit{teste\_update\_connects.sql} e \textit{teste\_update\_wears.sql}.

Os \textit{triggers} relativos a esta secção só diferem dos anteriores na medida em que são efectuados antes de um \textit{update} em vez de antes de um \textit{insert}. Assim, nos troços de código anteriores, onde se lê 

\textit{create trigger check\_overlap\_time\_period\_Patient\_PAN before insert on Wears} deve-se ler

 \textit{create trigger check\_overlap\_time\_period\_Patient\_PAN before update on Wears} e onde se lê
 
  \textit{create trigger check\_overlap\_time\_period\_Device\_PAN before insert on Connects} deve-se ler
  
   \textit{create trigger check\_overlap\_time\_period\_Device\_PAN before insert on Connects}.

No entanto, para esta secção foi utilizado um maior número de testes pois existe um maior número de casos de atualizações possíveis do que inserções.

Assim, a base de dados utilizada para os testes foi a presente no ficheiro \textit{database\_triggers\_update.sql}, sendo obtidas as seguintes tabelas:

\begin{figure}[ht!]
\centering
\includegraphics[scale=0.5]{update_connects.jpg}
\caption{Tabela \textit{Connects}}
\end{figure}

\begin{figure}[ht!]
\centering
\includegraphics[scale=0.5]{update_wears.jpg}
\caption{Tabela \textit{Wears}}
\end{figure}


Na tabela da figura 7, a linha que é atualizada é sempre a última, correspondente a um \textit{device} igual a pelo menos um que se encontra na tabela. Porém, possui um PAN diferente destes.

Na tabela da figura 8, as linhas que são atualizadas são a segunda e última linhas que testam,  respectivamente, dois pacientes possuirem a mesma PAN simultaneamente e o mesmo paciente ter duas PANs simultaneamente.
\section{\textit{SQL queries}}


\end{document}